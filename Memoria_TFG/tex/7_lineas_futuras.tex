\capitulo{7}{Lineas de trabajo futuras}

Este capítulo recoge mejoras y extensiones del sistema desarrollado. Se proponen líneas de trabajo orientadas a optimizar el modelo, ampliar el conjunto de datos y reforzar la validación clínica. Estas ideas servirán de base para futuras versiones.

\begin{enumerate}
    \item \textbf{Aumentar el conjunto de datos de entrenamiento:} En este proyecto se ha contado con un total de 34 ecografías, lo cual es una muestra limitada para entrenar un modelo de segmentación robusto y generalizable. Como linea futura, se plantea ampliar el dataset incorporando más imágenes provenientes de diferentes fuentes y con una mayor diversidad de patrones, como ecografías obtenidas en diversos equipos de ultrasonido o en distintas condiciones clínicas. Esto permitiría al modelo capturar mayor diversidad de patrones y mejorar su capacidad para generalizar a nuevos casos.
    \item \textbf{Incorporar la segmentación del cuarto ventrículo:} Inicialmente se exploró la posibilidad de incluir la segmentación del cuarto ventrículo en el modelo. Sin embargo, debido a su tamaño reducido y al desafío asociado a su segmentación precisa, la estructura presentó una precisión cercana a cero en las evaluaciones preliminares. Tras una valoración cuidadosa, se decidió priorizar las estructuras principales del cerebelo, que tienen una mayor relevancia clínica en el contexto del proyecto. Como línea futura, se podría reconsiderar su incorporación si se cuenta con técnicas avanzadas o de un dataset ampliado que mejore la precisión en este tipo de estructuras.
    \item \textbf{Añadir funcionalidades a la aplicación:} se plantea incorporar nuevas funcionalidades que enriquezcan el uso de la herramienta. Por ejemplo, se podría desarrollar una herramienta interactiva para ajustar o refinar los resultados según sea necesario. También sería posible incluir opciones para personalizar parámetros relacionados con el modelo o la visualización, ofreciendo un mayor control sobre el análisis. Además, se podría integrar un módulo de predicción diagnóstica basado en los resultados de segmentación, lo que permitiría extraer conclusiones clínicas automatizadas, aportando un valor significativo para los profesionales médicos y facilitando la toma de decisiones en entornos clínicos.
\end{enumerate}