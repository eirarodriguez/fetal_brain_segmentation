\apendice{Anexo de sostenibilización curricular}

\section{Introducción}
El proyecto se ha centrado en el desarrollo de una aplicación basada en inteligencia artificial para la segmentación automática de estructuras cerebrales fetales en imágenes ecográficas. Más allá del componente técnico, el proyecto ha supuesto una oportunidad para reflexionar sobre el papel que la tecnología puede desempeñar en la construcción de un sistema sanitario más justo, accesible y eficiente. En este capítulo se recoge cómo se han integrado los principios de sostenibilidad a lo largo del desarrollo del trabajo, siguiendo las directrices de la CRUE y en sintonía con los Objetivos de Desarrollo Sostenible (ODS) de la Agenda 2030.

\section{Tecnología para el bien común (ODS 3 y ODS 10)}
Uno de los pilares fundamentales de este proyecto ha sido la voluntad de mejorar la atención sanitaria mediante soluciones tecnológicas accesibles. Al desarrollar una aplicación capaz de asistir en el diagnóstico prenatal, se contribuye directamente al ODS 3: Salud y Bienestar, facilitando la detección precoz de posibles alteraciones en el desarrollo fetal. Esta detección, además, puede realizarse sin necesidad de equipamiento costoso o intervenciones invasivas, lo que la hace especialmente útil en contextos clínicos con recursos limitados.

Además, se promueve una sanidad más equitativa (ODS 10: Reducción de las desigualdades), ya que la inteligencia artificial puede reducir la variabilidad diagnóstica entre profesionales, democratizando el acceso a una segunda opinión médica automatizada, especialmente en zonas rurales o con escasez de especialistas.
\section{Eficiencia y sostenibilidad tecnológica (ODS 9 y ODS 12)}
Durante el desarrollo del proyecto, se han tomado decisiones orientadas a minimizar el consumo de recursos computacionales. Por ejemplo, el uso de técnicas como el early stopping permiten reducir el tiempo de entrenamiento y el uso de energía. Estas prácticas están alineadas con el ODS 12: Producción y consumo responsables y el ODS 9: Industria, innovación o infraestructura, promoviendo soluciones digitales sostenibles y eficientes.

Asimismo, se ha priorizado el uso de herramientas y librerías de código abierto (como PyTorch Lightning, OpenCV, Streamlit y COCO) fomentando la reutilización de software y reduciendo la dependencia de licencias comerciales, lo que además refuerza la accesibilidad para futuros desarrolladores o centros hospitalarios con recursos limitados.

\section{Educación, responsabilidad ética y desarrollo profesional (ODS 4 y ODS 17)}
Este trabajo me ha ayudado a desarrollar una conciencia ética respecto al uso de los datos médicos. He trabajado con conjuntos de datos sensibles, por lo que he tomado medidas para preservar la anonimización y asegurar un uso responsable de la información. Esta reflexión está directamente relacionada con el ODS 4: Educación de calidad, ya que he podido aplicar de forma transversal los principios éticos y de responsabilidad social aprendidos a lo largo del grado.

Además, al integrar el proyecto dentro de un ecosistema de trabajo reproducible, documentado y abierto a colaboraciones, se favorece el desarrollo de alianzas interdisciplinares (ODS 17: Alianzas para lograr los objetivos). La interfaz que he desarrollado de software contribuye a una integración real de la tecnología en el entorno médico.

\section{Conclusión}
A través de este trabajo, he podido comprobar que la sostenibilidad no se limita al ámbito ambiental, sino que también implica compromisos sociales, éticos y educativos. Integrar estos principios en un proyecto técnico ha supuesto un desafío enriquecedor y he reforzado mi convicción de que, como futura profesional de la ingeniería de la Salud, tengo la responsabilidad de aplicar mis conocimientos al servicio de una sociedad más justa, inclusiva y sostenible. En este sentido, el proyecto representa una pequeña aportación al compromiso colectivo con los ODS y con un desarrollo tecnológico humano.
