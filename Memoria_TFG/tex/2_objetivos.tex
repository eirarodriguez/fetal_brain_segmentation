\capitulo{2}{Objetivos}

Este proyecto tiene como propósito desarrollar un modelo de segmentación de imágenes ecográficas del cerebelo fetal, empleando técnicas de aprendizaje profundo, con la intención de apoyar el análisis médico y mejorar la precisión en el diagnóstico prenatal. A continuación, se detallan los objetivos generales y técnicos que guiarán el desarrollo del proyecto.

\section{Objetivos generales}
\begin{itemize}
    \item Desarrollo de un sistema de segmentación de imágenes ecográficas 2D de cerebelo fetal utilizando técnicas de \textit{Deep Learning}, con el objetivo de apoyar el análisis médico.
    \item Evaluar la precisión y fiabilidad del sistema desarrollado mediante la comparación de distintas arquitecturas y su validación frente a segmentaciones manuales realizadas por el autor, bajo supervisión y aprobación de un experto médico.
    \item Crear una aplicación web que facilite el uso del sistema de segmentación por parte de los profesionales médicos.
\end{itemize}
\section{Objetivos técnicos}
\begin{itemize}
    \item Implementar un sistema de segmentación basado en redes neuronales convolucionales, explorando diferentes arquitecturas dentro del entorno de desarrollo de \textit{Google Colab}.
    \item Aplicar técnicas de preprocesamiento a las imágenes ecográficas 2D para mejorar el desempeño del modelo y reducir el riesgo de sobreajuste.
    \item Validar el sistema utilizando métricas específicas como \texttt{IoU (Intersection over Union)} y realizar ajustes iterativos de los hiperparámetros para optimizar su rendimiento.
    \item Diseñar e implementar una interfaz web funcional que permita cargar imágenes ecográficas y visualizar de forma clara los resultados de segmentación.
    \item Documentar el progreso del desarrollo utilizando repositorios en plataformas como GitHub, favoreciendo la organización del trabajo.
\end{itemize}
\section{Objetivos personales}
Los objetivos a nivel personal son los siguientes:
\begin{itemize}
    \item Profundizar en los conocimientos prácticos adquiridos en el grado (Grado en Ingeniería de la Salud) sobre las técnicas de \textit{Deep Learning}, especialmente aquellas aplicadas a la segmentación de imágenes médicas.
    \item Aprender sobre diferentes modelos de \textit{Deep Learning}, métricas de evaluación y factores que influyen en el proceso de entrenamiento, como los hiperparámetros o la calidad del conjunto de datos.
    \item Familiarizarme con el entorno de desarrollo de \texttt{PyTorch}, aprendiendo a implementar, entrenar y evaluar modelos de segmentación semántica en ecografías 2D.
    \item Adquirir conocimientos sobre la anatomía del cerebelo fetal y comprender la importancia clínica del desarrollo cerebeloso durante el embarazo.
    \item Desarrollar y mejorar la capacidad de documentar y comunicar de forma clara y comprensible los procesos técnicos y resultados obtenidos.
\end{itemize}









